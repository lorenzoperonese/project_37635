\documentclass{article}
\usepackage[utf8]{inputenc}
\usepackage[italian]{babel}
\usepackage{datetime}
\selectlanguage{italian}
\usepackage{algorithm}
\usepackage{algpseudocode}
\usepackage{siunitx}
\usepackage{changepage}
\usepackage{tabularx}
\usepackage{enumitem}
\usepackage{array}
\usepackage{hyperref}
\hypersetup{
  colorlinks,
  linkcolor=black,
  urlcolor=blue
}
\newcolumntype{C}{>{\centering\arraybackslash}X}

\newdate{date}{3}{09}{2023}
\title{\textsc{Kebabo}\\
IA per il gioco Connect X\\
\large Corso di Algoritmi e Strutture di Dati}
\author{
  Lorenzo Peronese (0001081726),
  Omar Ayache (0001068895)
}
\date{
	Alma Mater Studiorum \\
    Bologna \\
  \displaydate{date}
}

\begin{document}
\maketitle
\thispagestyle{empty}
\pagebreak
\pagenumbering{roman}
\tableofcontents
\pagebreak

\pagenumbering{arabic}
\section{Introduzione}

\textsc{Connect X} è una variante del celebre gioco \textsc{Connect 4 (Forza 4)}, in cui le
dimensioni della matrice di gioco e il numero di pedine da collegare variano; 
Connect 4 è infatti una configurazione di Connect X, la \textsc{6 7 4}.

A turno due giocatori fanno cadere delle pedine lungo le colonne della 
matrice di gioco, con l'obiettivo di allinearne in orizontale, 
verticale o diagonale un certo numero; quello che lo differenzia 
dal gioco \textsc{Tic-Tac-Toe (Tris)} è la gravità: una pedina messa in una certa 
colonna della matrice va ad occupare la posizione libera più in basso.

Connect 4 è un gioco risolto, ovvero esiste una sequenza di mosse che 
porta sempre e inevitabilmente alla vittoria di uno dei giocatori oppure a una patta; ovviamente,
Connect X non può esserlo a causa della grande varietà di configurazioni diverse; 
alcune di esse sono però risolte, come ad esempio le board \textsc{4 4 4} e \textsc{4 6 4} che,
giocate perfettamente da entrambi i giocatori, porteranno sempre rispettivamente a un
pareggio e a una vittoria del secondo giocatore.

Il progetto richiedeva lo sviluppo di un algoritmo in grado di giocare 
alcune configurazioni e selezionare sempre la mossa migliore entro un certo tempo limite; 
l'algoritmo avrebbe poi giocato due partite, una come primo giocatore e una come secondo,
per ogni board, contro ogni altro player; una vittoria (regolare o per errore avversario) 
porta \textsc{3 punti}, un pareggio \textsc{1 punto} e una sconfitta 
\textsc{0 punti}.
Le configurazioni sono riportate nella pagina seguente.

I due errori più comuni che terminano immediatamente la partita sono \textsc{timeout}, 
ovvero la scadenza del tempo per fare una mossa, e \textsc{mossa non legale}.

\'E stata poi fornita dai docenti l'interfaccia di gioco \textsc{CXGame}, che si occupa di 
inizializzare la partita, sia in modalità testuale, sia grafica; l'implementazione deve
essere fornita come \textsc{package}, e la classe che implementa l'interfaccia \textsc{CXPlayer}
deve contenere 3 metodi: \textsc{initPlayer}, \textsc{selectColumn} e \textsc{playerName}. 

L'algoritmo adottato ha diverse ottimizzazioni che sono state aggiunte progressivamente; ogni
qual volta veniva implementata una nuova versione, questa veniva testata contro la precedente
per verificarne la bontà; alcune implementazioni sono state mantenute
perchè migliori delle precedenti, mentre altre sono state scartate perchè non apportavano
un sigificativo miglioramento.

Per questo progetto è inoltre stato fatto uso della potenza di calcolo delle macchine di
laboratorio, per il testing delle varie versioni del player.
\pagebreak

\subsection{Configurazioni di gioco}
\vspace{20pt}
{
    \centering
    \begin{tabular}{| l | l | l | l |}
    \hline
    M & N & X \\ 
    \hline
    4 & 4 & 4 \\ 
    5 & 4 & 4 \\ 
    6 & 4 & 4 \\ 
    7 & 4 & 4 \\ 
    4 & 5 & 4 \\ 
    5 & 5 & 4 \\ 
    6 & 5 & 4 \\ 
    7 & 5 & 4 \\ 
    4 & 6 & 4 \\ 
    5 & 6 & 4 \\ 
    6 & 6 & 4 \\ 
    7 & 6 & 4 \\ 
    4 & 7 & 4 \\ 
    5 & 7 & 4 \\ 
    6 & 7 & 4 \\ 
    7 & 7 & 4 \\ 
    5 & 4 & 5 \\ 
    6 & 4 & 5 \\ 
    7 & 4 & 5 \\ 
    4 & 5 & 5 \\ 
    5 & 5 & 5 \\ 
    6 & 5 & 5 \\ 
    7 & 5 & 5 \\ 
    4 & 6 & 5 \\ 
    5 & 6 & 5 \\ 
    6 & 6 & 5 \\ 
    7 & 6 & 5 \\ 
    4 & 7 & 5 \\ 
    5 & 7 & 5 \\ 
    6 & 7 & 5 \\ 
    7 & 7 & 5 \\ 
    20 & 20 & 10 \\
    30 & 30 & 10 \\
    40 & 40 & 10 \\
    50 & 50 & 10 \\
    \hline
\end{tabular}\par
}
\pagebreak

\section{La nostra implementazione}

La versione definitiva del giocatore utilizza l'algoritmo \textsc{MiniMax} con 
potatura \textsc{Alpha-Beta} per esplorare l'albero delle mosse possibili.
Per aumentare i tagli di AlphaBeta, è stato aggiunto un semplice \textsc{move explorator
order}, che garantisce spesso di trovare la mossa migliore tra le prime analizzate; inoltre
fa uso di una \textsc{tabella delle trasposizioni} per memorizzare il
punteggio delle posizioni che vengono incontrate. 

La ricerca \textsc{Iterative Deepening}
permette poi di effettuare una visita a livelli dell'albero, aumentando gradualmente
la profondità e garantendo così la massima efficienza possibile.

Nelle board più grandi, questo non è sufficiente per esplorare l'intero albero di gioco,
perciò si rende necessaria una \textsc{funzione di valutazione} dei nodi non terminali, che
si occupa di analizzare la posizione e assegnare un punteggio in base alle proprie 
minacce e quelle dell'avversario.

\subsection{MiniMax-AlphaBeta}

Essendo Connect X un \textsc{gioco a somma zero}, ovvero la somma dei punteggio dei due giocatori
è sempre uguale a zero, il punto di partenza è stato l'algoritmo \textsc{MiniMax} studiato a lezione. 
Questo algoritmo permette di minimizzare la massima perdita possibile e massimizzare il minimo 
guadagno: l'algoritmo cerca la mossa migliore dal fondo dell'albero e risale fino alla posizione 
attuale, supponendo che entrambi i giocatori facciano sempre le scelte migliori.

Se l'algoritmo riesce ad esplorare l'intero Game Tree, sappiamo per certo che sceglierà la mossa 
migliore, ma purtroppo, soprattutto con configurazioni grandi, è molto difficile che questo accada; 
infatti, in una board $m \cdot n$, ogni cella può assumere 3 valori (\verb!P1, P2, FREE!) e quindi 
il numero di nodi dell'albero di gioco è pari a $3^{nm}$, anche se il valore reale è molto minore
perchè molti nodi sono posizioni illegali a causa della gravità.

La \textsc{potatura AlphaBeta} permette di tagliare alcuni rami dell'albero di gioco che non sono 
interessanti: ad esempio se il giocatore A massimizza e la prima mossa gli da un certo punteggio, se
il primo nodo figlio della seconda mossa ha un punteggio minore della mossa 1, sappiamo per certo che
la mossa 2 avrà uno score minore di quello del primo figlio (perchè il giocatore B minimizza), 
perciò non ha senso finire l'esplorazione di quel ramo perchè la mossa ad esso collegato non verrà 
mai scelta, e possiamo quindi potarlo.

Questa miglioria incide sulle prestazioni sulla base di un fattore: \textsc{l'ordinamento delle mosse}.
Visitando prima le mosse che si riveleranno poi migliori, il numero dei tagli sarà molto maggiore,
mentre con l'ordinamento opposto non ci sarà nessun taglio e il costo computazionale sarà uguale a 
quello di MiniMax. Infatti, ipotizzando un ordinamento delle mosse ottimale, AlphaBeta offre uno
speed-up quadratico rispetto a MiniMax.
\pagebreak

\begin{algorithm}[H]
    \caption{\textsc{MiniMax-AlphaBeta}}
    \label{alg:minimax}
    \begin{algorithmic}[H]
      \Procedure {alphaBetaSearch}{$board$, $depth$}
        \State $bestMove \gets -1$
        \For{$move$ \textsc{in} $\Call{possibleMoves}$}
            \State $score \gets \Call{AlphaBetaMin}{board,-\infty,\infty,depth}$
            \If{$score > alpha$}
              \State $alpha \gets score$
              \State $bestMove \gets move$
            \EndIf
      \EndFor
      \State \Return $bestMove$
      \EndProcedure \\
      \Procedure {AlphaBetaMax}{$board$,$alpha$,$beta$,$depth$}
        \If{$depth = 0$ \textbf{or} node is a leaf}
          \State \Return \Call{Evaluate}{$board$}
        \EndIf
        \For{$move$ \textsc{in} $\Call{possibleMoves}$}
            \State $score \gets \Call{AlphaBetaMin}{board,alpha,beta,depth-1}$
            \If{$score \geq beta$}
            \Comment potatura AlphaBeta
              \State \Return $beta$
            \EndIf
            \If{$score > alpha$}
              \State $alpha \gets score$
            \EndIf
      \EndFor
      \State \Return $alpha$
      \EndProcedure \\
      \Procedure {AlphaBetaMin}{$board$,$alpha$,$beta$,$depth$}
        \If{$depth = 0$ \textbf{or} node is a leaf}
          \State \Return \Call{Evaluate}{$board$}
        \EndIf
        \For{$move$ \textsc{in} $\Call{possibleMoves}$}
            \State $score \gets \Call{AlphaBetaMax}{board,alpha,beta,depth-1}$
            \If{$score \leq alpha$}
            \Comment potatura AlphaBeta
              \State \Return $alpha$
            \EndIf
            \If{$score < beta$}
              \State $beta \gets score$
            \EndIf
      \EndFor
      \State \Return $beta$
      \EndProcedure
    \end{algorithmic}
  \end{algorithm}

\pagebreak


\subsection{Move explorator order}
Per usufruire al massimo dei vantaggi del pruning di AlphaBeta, abbiamo utilizzato un array che contiene
l'ordine con cui esplorare le mosse, con l'obiettivo di iniziare dalle mosse più promettenti.
L'idea che sta alla base è che le colonne centrali sono quelle più strategiche e da cui passano
più collegamenti, perciò è spesso vero che le mosse migliori sono quelle vicino al centro.

L'array viene riempito in questa maniera:

$moveOrder[i] = columns / 2 + (1 - 2 \cdot (i $ mod $ 2)) \cdot (i + 1) / 2, \forall i \in \{0,columns-1\} $
Il risultato sarà questo: 
\begin{itemize}
  \item $moveOrder[0] \gets columns/2$
  \item $moveOrder[1] \gets columns/2 -1$
  \item $moveOrder[2] \gets columns/2 +1$
  \item ...
  \item $moveOrder[columns-2] \gets 0$
  \item $moveOrder[columns-1] \gets columns-1$
\end{itemize}

\subsection{Tabella delle trasposizioni}

Nell'analizizzare tutti i possibili stati di una partita, MiniMax può trovare più nodi con la stessa
posizione delle pedine, ma raggiunta con un ordine di mosse differente (\emph{trasposizione} 
in gergo scacchistico). La tabella delle trasposizioni viene adottata proprio per evitare di sprecare 
tempo a ricalcolare il punteggio della trasposizione di una mossa già calcolata.

La struttura dati \textsc{HashMap} di Java memorizza score e profondità di una posizione; la funzione di 
hash è la \textsc{Zobrist Hashing}, che associa ad ogni cella della matrice 3 interi casuali, uno per 
ogni possibile valore che può assumere (\verb!P1, P2, FREE!). L'hash della posizione viene calcolato 
facendo la \verb!XOR! di tutti i valori Zobrist associati alla matrice di gioco, in base alla 
posizione raggiunta.

Quando viene chiamato MiniMax, prima di iniziare ad analizzare le mosse possibili viene controllato
in tempo lineare se la posizione da calcolare si trova già nella HashMap e se la profondità con cui è
stata calcolata è maggiore o uguale a quella richiesta; se è presente, viene ritornato direttamente
quel valore, altrimenti MiniMax calcola il punteggio, che viene poi salvato nella tabella.

\pagebreak

\subsection{Iterative Deepening}

L'algoritmo MiniMax visita l'albero in profondità, in post-ordine; se non è in grado di visitare
l'intero Game Tree entro il tempo limite, allora non è riuscito ad analizzare le ultime mosse possibili,
rischiando così di sbagliare l'analisi e restituire una mossa non ottimale.

\'E di vitale importanza che il giocatore abbia analizzato ogni mossa e abbia una visione totale della 
board, anche se questo significa perdere un po' in profondità; questo problema viene risolto
dall'algoritmo \textsc{IterativeDeepening}, che implementa una visita a livelli dell'albero di gioco.

Grazie a questa ottimizzazione, viene calcolata la mossa migliore a profondità crescenti e viene
restituito l'ultimo valore calcolato prima dello scadere del tempo.

In realtà, questo processo avviene solamente durante la prima mossa della partita, mentre nelle successive
la profondità non viene più inizializzata a $0$, ma alla profondità raggiunta durante la mossa
precedente (-2 per essere sicuri di avere una mossa valida da ritornare).

Inoltre, se al crescere della profondità il giocatore si accorge che non ha modo di evitare la sconfitta,
gioca l'ultima mossa migliore valida per cercare di prolungare il più possibile la partita, sperando
in un errore avversario.

Nonostante il costo computazionale dell'algoritmo sia uguale ad AlphaBeta, perchè quello che incide
è il costo dell'ultima profondità cercata, IterativeDeepening svolge più operazioni di ricerca a
diverse profondità e pertanto il suo costo computazionale ha delle costanti moltiplicative che
nella pratica lo rendono complessivamente più lento.

\begin{algorithm}[H]
  \caption{\textsc{Iterative Deepening}}
  \label{alg:itdeepening}
  \begin{algorithmic}
    \Procedure {IterativeDeepening}{$board$}
      \State $bestmove \gets -1$
      \State $depth \gets 0$
      \While{$\Call{thereIsTime}$}
        \State $depth \gets depth + 2$
        \State $bestMove \gets \Call{alphaBetaSearch}{board,depth}$
      \EndWhile
      \State \Return $bestMove$
    \EndProcedure
  \end{algorithmic}
\end{algorithm}

\pagebreak

\subsection{Euristica}

Alla scadenza del tempo, l'algoritmo lancia un'eccezione che blocca tutte le chiamate ricorsive e 
deve valutare i nodi raggiunti: se questi sono foglie, la valutazione è triviale: \textsc{Integer.MAXVALUE}
se quella serie di mosse gli permette di vincere, \textsc{Integer.MINVALUE} se lo portano a una sconfitta
e $0$ se la partita si conclude con un pareggio.

Se però la partita non si conclude con l'ultima mossa, è necessario poter confrontare diverse posizioni
e stabilire la migliore, assegnando ad ognuna di esse un punteggio.

La strategia che abbiamo adottato è stata quella di contare le serie aperte di pedine, ovvero quelle
che in futuro potrebbero portare alla vittoria, e dare un punteggio pari al numero di pedine collegate
elevato al quadrato. Lo score della posizione è dato dunque dalla differenza tra il punteggio delle serie
aperte del giocatore e quelle dell'avversario.

Più nel dettaglio, ogni riga di lunghezza $n$ viene divisa in $n - x +1$ sottogruppi ordinati di lunghezza $x$ 
(dove $x$ è il numero di simboli da allineare), come da schema:

\vspace{10pt}
{\centering
{\noindent
\begin{tabularx}{7cm}{@{}|C|C|C|C|C|C|C|}
  \hline
  {1} & {2} & {3} & {4} & {5} & {6} & {7}\\
  \hline
  \end{tabularx} \par
}
}
\vspace{5pt}
\begin{adjustwidth}{2.56cm}{}
  \begin{tabularx}{4cm}{|C|C|C|C|}
  \hline
  1 & 2 & 3 & 4 \\
  \hline
  \end{tabularx}
  \end{adjustwidth}

\begin{adjustwidth}{3.56cm}{}
  \begin{tabularx}{4cm}{|C|C|C|C|}
  \hline
  2 & 3 & 4 & 5 \\
  \hline
  \end{tabularx}
  \end{adjustwidth}

\begin{adjustwidth}{4.56cm}{}
  \begin{tabularx}{4cm}{|C|C|C|C|}
  \hline
  3 & 4 & 5 & 6 \\
  \hline
  \end{tabularx}
  \end{adjustwidth}

\begin{adjustwidth}{5.56cm}{}
  \begin{tabularx}{4cm}{|C|C|C|C|}
  \hline
  4 & 5 & 6 & 7 \\
  \hline
  \end{tabularx}
  \end{adjustwidth}

\vspace{10pt}
Se in un sottogruppo ci sono pedine di entrambi i giocatori oppure le celle sono tutte vuote, allora il suo punteggio
sarà $0$. Altrimenti, se ci sono solamente pedine di uno dei due giocatori e celle libere, il punteggio sarà dato dal
numero di pedine al quadrato.

Il metodo di assegnamento dello score di diagonali e anti-diagonali è molto simile a questo.

Invece, per ogni colonna viene presa la cella non vuota più in alto
e vengono contate le pedine consecutive dello stesso colore verso il basso; se il numero delle pedine consecutive,
sommato al numero delle celle vuote soprastanti è maggiore o uguale al numero di simboli che bisogna allineare per vincere,
allora quella serie sarà aperta e il punteggio verrà aggiornato di conseguenza.

Oltre a questo, un'altra idea è stata quella di dare un numero di punti ad ogni cella della matrice
in base alla posizione strategica, quindi più punti alle celle centrali e meno a quelle più esterne.

Una volta implementata la matrice di punteggi, il cambiamento non si è notato particolarmente perchè
di per sè già la funzione di valutazione include questo tema: infatti le celle centrali fanno parte di
diverse serie che possono portare alla vittoria, al contrario di quelle più esterne, quindi controllarle porta
più punti; pertanto abbiamo deciso di mantenere la funzione originale ed eliminare questa parte.

\begin{algorithm}[H]
  \caption{\textsc{Euristica}}
  \label{alg:evaluate}
  \begin{algorithmic}
    \Procedure {evaluate}{$board$}
      \If{$\Call{myWin}$}
        \State \Return $\infty$
      \ElsIf{$\Call{yourWin}$}
        \State \Return $-\infty$
      \ElsIf{$\Call{isDraw}$}
        \State \Return $0$
      \Else
        \State \Return $\Call{diagonalScore}{board} + \Call{antiDiagonalScore}{board} + 
        \Call{rowScore}{board} + \Call{columnScore}{board}$
      \EndIf
    \EndProcedure
  \end{algorithmic}
\end{algorithm}

\section{Conclusioni}
A causa dell'ordinamento della valutazione delle mosse non sempre ottimale, il costo 
computazionale dell'algoritmo è compreso fra il costo di \textsc{MiniMax} O$(3^{nm})$ (ordinamento delle mosse
pessimo) e quello di \textsc{AlphaBeta} O$({\sqrt{3^{nm}}})$ (ordinamento delle mosse ottimo). 

A questo, va aggiunto il costo di \textsc{IterativeDeepening}, che però è di ordine inferiore e quindi 
trascurabile a livello asintotico, e il costo della \textsc{valutazione euristica}; questa viene effettuata 
per ogni nodo raggiunto (talvolta più volte per lo stesso nodo a profondità diverse) e il costo di una singola 
valutazione è pari a $\Theta(mnx)$, con $m$ e $n$ numero di righe e colonne della matrice e $x$ numero di simboli
da allineare per vincere.

Infine, il costo di hashing per la \textsc{tabella delle trasposizioni} è costante.

\section{Fonti}

Per lo sviluppo del progetto abbiamo consultato i siti web 
\href{https://www.chessprogramming.org}{chessprogramming.org}, che contiene diverse informazioni riguardo alla programmazione 
dei motori scacchistici e \href{http://blog.gamesolver.org}{blog.gamesolver.org}, 
il progetto di Pascal Pons per risolvere il gioco Connect 4.

\end{document}
